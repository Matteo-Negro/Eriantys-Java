% !TeX encoding = UTF-8
% !TeX spellcheck = it_IT

\documentclass[a4paper]{article}

\pdfminorversion=7
\pdfcompresslevel=9
\pdfobjcompresslevel=2

\usepackage[hidelinks]{hyperref}
\usepackage[margin=60pt]{geometry}
\usepackage[T1]{fontenc}
\usepackage[utf8]{inputenc}
\usepackage[italian]{babel}

\setlength{\parindent}{0pt}

\title{\textbf{Peer-Review 1: UML}}
\author{Riccardo Milici, Riccardo Motta, Matteo Negro\\Gruppo 2}
\date{A.A. 2021 - 2022}

\hypersetup{
	pdftitle={Peer-Review 1: UML},
	pdfauthor={Riccardo Milici, Riccardo Motta, Matteo Negro}
}

\begin{document}

	\maketitle
	
	Valutazione del diagramma UML delle classi del gruppo 1.
	
	\section{Aspetti positivi}
	
	Indicare in questa sezione quali sono secondo voi i lati positivi dell'UML dell'altro gruppo. Se avete qualche difficoltà, provate a simulare il gioco a mano, immaginandovi quali sono le invocazioni di metodo che avvengono in certe situazioni che vi sembrano importanti (ad esempio, la fusione delle isole oppure il calcolo dell'influenza).
	
	\section{Aspetti negativi}
	
	Come nella sezione precedente, indicare quali sono secondo voi i lati negativi.
	
	\section{Confronto tra le architetture}
	
	Individuate i punti di forza dell'architettura dell'altro gruppo rispetto alla vostra, e quali sono le modifiche che potete fare alla vostra architettura per migliorarla.

\end{document}
